%-------------------------------------------------------------------------------
%	SECTION TITLE
%-------------------------------------------------------------------------------
\vspace{-1mm}
\cvsection{Projects}


%-------------------------------------------------------------------------------
%	CONTENT
%-------------------------------------------------------------------------------
\begin{cventries}
%---------------------------------------------------------
 \cventry
    {Researcher and network engineer} % Role
    {mtunzi: Software-defined Network Access Control for Small Networks} % Organization
    {\href{https://github.com/innoobijr/cse-550-finalproject-mtunzi}{GitHub Repository}} % GitHub
    {September 2021 - January 2022} % Date(s)
    {
      \begin{cvitems} % Description(s) of tasks/responsibilities
        \item {
          \textit{mtunzi} is a system for software-defined network access control and captive networking aimed toward the networking needs of small WISPs and community network operators. The system uses Openflow as a communication protocol and freeRADIUS server authentication process, both run on Mininet (i.e. network emulator).
            The captive portal uses flask application.
        }
        \item {
            Collaboratively design the flask app portal and software-defined approach to register and authenticate switches, authenticate hosts \& users, and dynamically manage hosts \& users.
        }
      \end{cvitems}
    }
%---------------------------------------------------------
 \cventry
    {Project Leader and Database Analyst} % Role
    {Microsoft Malware Detection for UW Informatics x Microsoft Corp.} % Organization
    {} % GitHub
    {January 2020 - March 2021} % Date(s)
    {
      \begin{cvitems} % Description(s) of tasks/responsibilities
        \item {
          Lead the team to build normalized database model, drafting the entity relationship diagram (ERD), and investigate the top 5 devices, characteristics that are prone to get infected by analyzing over eight million rows of real data representing the malware-infected devices.
        }
        \item {
          Collaboratively design normalized and highly optimized database containing only business related to meet the needs using advanced SQL and high-level database management. Microsoft SQL Server is the main software. The data analysis was done in Python.
        }
      \end{cvitems}
    }
%---------------------------------------------------------
 \cventry
    {Full-Stack Mobile Application Developer} % Role
    {PathDeck: COVID-19 Health Tracking Application} % Organization
    % {\href{https://github.com/nussarafirn/pathdeck}{github.com/nussarafirn/pathdeck}} % GitHub
    {\href{https://github.com/nussarafirn/pathdeck}{GitHub Repository}} % GitHub
    {June 2020 - August 2020} % Date
    {
      \begin{cvitems} % Description(s) of tasks/responsibilities
        \item {
          Is an application allows users to track their health, locations, activities, and notify people in contact quickly if they get infected. However, this application can also serve as a note taking or daily planner application. More information about features please visit the GitHub repository. 
        }
        \item {
          Contributed mainly as a full-stack developer together with Kieu Trihn using Android environment, Postgres, Heroku, Figma, SQL, JavaScript and Java.The demo video can be found at \href{https://youtu.be/oJ3nPrunM60}{youtu.be/oJ3nPrunM60 
        }}
      \end{cvitems}
    }
%---------------------------------------------------------

   \cventry
    {Front-end Developer} % Role
    {NutriBites: On-Demand Campus Dining Service at DubHacks 2019} % Organization
    % {\href{https://github.com/nussarafirn/Dubhacks2019}{github.com/nussarafirn/Dubhacks2019}} % GitHub
     {\href{https://github.com/nussarafirn/Dubhacks2019}{GitHub Repository}} % GitHub
    {October 2019 - October 2019} % Date(s)
    {
      \begin{cvitems} % Description(s) of tasks/responsibilities
        \item {
         A software prototype for tight-schedule students to pre-order campus nutritious food online. The features include utilizing the calendar API and ML to automatically ordering lunch based on availability, behavioral analysis of orders, and coin-earning system from consuming healthier food or delivering for their schoolmates as an incentive.
        }
        \item {
          Contributed on the prototype of platform collaboratively from brainstorming ways to differentiate NutriBites from other student-run food services and developed front-end using Bootstrap, HTML, CSS, and JavaScript.}
      \end{cvitems}
    }
%---------------------------------------------------------

   \cventry
    {Project Leader and Data Visualization developer} % Role
    {Chicago Crime Data Visualization} % Organization
    % {\href{https://github.com/nussarafirn/info-201-final-project}{github.com/nussarafirn/info-201-final-project}} % GitHub
     {\href{https://github.com/nussarafirn/info-201-final-project}{GitHub Repository}} % GitHub
    {June 2019 - August 2019} % Date(s)
    {
      \begin{cvitems} % Description(s) of tasks/responsibilities
        \item {
          Is a shiny application for interactive data visualization using crime datasets in Chicago over the past 10 years to inform locals and tourists about the locations and areas, crime types, and timeline that crimes were committed.
        }
        % \item {The analysis is presented by various types of interactive data visualizations including geographical map shows the locations crime committed, bar chart represents the top ten most committed types of crime in Chicago over the past 10 years, and heatmap showing crime distribution over the different time periods and areas.}
        \item {
            Designed and implemented mainly on the interactive heatmap showing crime distribution in different time/area using data exploration and visualization skills in advanced R and HTML/CSS language. }
        % \item {
        %   Led the team discussion from discovering topics and suitable visualization types, delegated tasks, followed up for updates, and assisted teammates on debugging.
        % }
      \end{cvitems}
    }
%--------------------------------------------------------
%---------------------------------------------------------
\end{cventries}
